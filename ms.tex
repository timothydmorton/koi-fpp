%% This is emulateapj reformatting of the AASTEX sample document
%%
\documentclass[preprint2]{aastex}
%\documentclass[iop]{emulateapj}

\usepackage{natbib}
\usepackage{acronym}
\bibliographystyle{apj}

\newcommand{\vdag}{(v)^\dagger}
\newcommand{\myemail}{tdm@astro.princeton.edu}

\newcommand{\ntotal}{5500}
\newcommand{\nfail}{1000}
\newcommand{\nval}{1500}
\newcommand{\nvalnew}{1000}
\newcommand{\nfp}{500}
\newcommand{\nfpnew}{200}
\newcommand{\kepler}{\textit{Kepler}}
\newcommand{\vespa}{\texttt{vespa}}
\newcommand{\isochrones}{\texttt{isochrones}}

\defcitealias{Morton:2012}{M12}

\acrodef{fpp}[FPP]{false positive probability}
\acrodef{koi}[KOI]{\kepler\ Object of Interest}
\acrodefplural{koi}[KOIs]{\kepler\ Objects of Interest}
\acrodef{nexsci}[NExScI]{NASA Exoplanet Science Institute}

%% You can insert a short comment on the title page using the command below.

\slugcomment{}

\shorttitle{False Positive Probabilities for KOIs}
\shortauthors{Morton et al.}

%% This is the end of the preamble.  Indicate the beginning of the
%% paper itself with \begin{document}.

\begin{document}

%% LaTeX will automatically break titles if they run longer than
%% one line. However, you may use \\ to force a line break if
%% you desire.

\title{False Positive Probabilities for all Kepler Objects of Interest: \\
        \nvalnew\ newly validated planets and \nfpnew\ likely false positives}

%% Use \author, \affil, and the \and command to format
%% author and affiliation information.
%% Note that \email has replaced the old \authoremail command
%% from AASTeX v4.0. You can use \email to mark an email address
%% anywhere in the paper, not just in the front matter.
%% As in the title, use \\ to force line breaks.

\author{Timothy D. Morton}
\affil{Department of Astrophysical Sciences, Princeton University}

%\altaffiltext{1}{Department of Astrophysics, Princeton University}

%% Mark off your abstract in the ``abstract'' environment. In the manuscript
%% style, abstract will output a Received/Accepted line after the
%% title and affiliation information. No date will appear since the author
%% does not have this information. The dates will be filled in by the
%% editorial office after submission.

\begin{abstract}
We present the results of applying a fully automated transit signal
false positive probability calculating procedure to every
\kepler\ Object of Interest (KOI) from the Q1-Q16 catalog release.
Out of \ntotal\ KOIs, we determine that \nval\ have probabilities
$<$1\% to be astrophysical false positives, and thus may be considered
validated planets.  \nvalnew\ of these have not yet been validated or
confirmed by other methods.  In addition, we identify \nfp\ KOIs
likely to be false positives ($>$90\% probability), \nfpnew\ of which
have not yet been identified as such. A side product of these
calculations is full stellar property posterior samplings for every
host star, modeled as single, binary, and triple.  These calculations
use \vespa, a publicly available Python package able to be easily
applied to any transiting exoplanet candidate.
\end{abstract}

%% Keywords should appear after the \end{abstract} command. The uncommented
%% example has been keyed in ApJ style. See the instructions to authors
%% for the journal to which you are submitting your paper to determine
%% what keyword punctuation is appropriate.

%% Authors who wish to have the most important objects in their paper
%% linked in the electronic edition to a data center may do so in the
%% subject header.  Objects should be in the appropriate "individual"
%% headers (e.g. quasars: individual, stars: individual, etc.) with the
%% additional provision that the total number of headers, including each
%% individual object, not exceed six.  The \objectname{} macro, and its
%% alias \object{}, is used to mark each object.  The macro takes the object
%% name as its primary argument.  This name will appear in the paper
%% and serve as the link's anchor in the electronic edition if the name
%% is recognized by the data centers.  The macro also takes an optional
%% argument in parentheses in cases where the data center identification
%% differs from what is to be printed in the paper.

\keywords{}


\section{Introduction}

The \kepler\ mission has revolutionized our understanding of
exoplanets.  Among many other important discoveries, \kepler\ has
identified several previously unsuspected features of planetary
systems, such as the prevalence of planets between the size of Earth
and Neptune, and a population of very compact multiple-planet
systems. And perhaps most notably, it has enabled for the first time
estimates of the occurrence rates of small planets ($\gtrsim$1
$R_\oplus$) out to orbits of about one year.  It is important to
remember, however, that these revolutionary discoveries depend
intimately on another revolution---how to interpret transiting planet
\textit{candidate} signals in the absence of unambiguous positive
confirmation of their veracity.

Before \kepler, every survey searching for transiting exoplanets had a 

\ac{fpp} and then \ac{fpp}.

%%%%%%%%%%%%%%%%%%%%%%%%%%%%%%%%%%%%%%%%%%%%%%%%%%%%%%

\section{Methods}
\label{sec:methods}

In this work, we apply the fully automated \ac{fpp}-computing
procedure described in \citet{Morton:2012}
\citepalias[][hereafter]{Morton:2012} to all \acp{koi} listed in the
``cumulative'' \kepler\ data table at the \ac{nexsci} Exoplanet
Archive.  While we refer the reader to \citetalias{Morton:2012} for a
detailed description of the method, we outline it briefly here, noting
especially the differences between the current implementation and
\citetalias{Morton:2012}.  The procedure as implemented in this work
is now publicly available in the Python module
\vespa\footnote{\url{https://github.com/timothydmorton/vespa}}.

The basic goal of \vespa\ is to assign probabilities to different
astrophysical hypotheses that might describe a transiting planet
candidate signal.  If $\{H_i\}$ is the set of all considered
hypotheses, the probability for any given model $i$ is
\begin{equation}
  \label{eq:prob}
  \mathrm{Pr}(H_i) = \frac{\mathcal L_i \pi_i}{\displaystyle \sum_j \mathcal L_j \pi_j}
\end{equation}
where $\mathcal L_i$ and $\pi_i$ are the ``likelihood'' and ``prior''
factors for each hypothesis.



eclipse signals as simple
trapezoidal 



%%%%%%%%%%%%%%%%%%%%%%%%%%%%%%%

\subsection{Stellar Properties}
\label{sec:methods:stellar}

The first step in calculating the \ac{fpp} of a transit signal is to
determine the physical properties of the signal host star.  To
accomplish this, we use the Python module \isochrones
\citep{isochrones}, which fits a stellar model grid of choice to a set
of observed properties of a given star.  

In \citetalias{Morton:2012}, 

%%%%%%%%%%%%%%%%%%%%%%%%%%%%%%%

\subsection{False Positive Probabilities}
\label{sec:methods:fpp}


%%%%%%%%%%%%%%%%%%%%%%%%%%%%%%%%%%%%%%%%%%%%%%%%%%%%%%

\section{Results}
\label{sec:results}

%%%%%%%%%%%%%%%%%%%%%%%%%%%%%%%

\subsection{Stellar Properties}
\label{sec:results:stars}

%%%%%%%%%%%%%%%%%%%%%%%%%%%%%%%

\subsection{False Positive Probabilities}
\label{sec:results:fpp}

%%%%%%%%%%%%%%%%%%%%%%%%%%%%%%%%%%%%%%%%%%%%%%%%%%%%%%

\section{Conclusions}
\label{sec:conclusions}

\acknowledgments
It is a pleasure to thank
\ldots\

\clearpage
\bibliography{ms}
\clearpage


\end{document}

%%
%% End of file `sample.tex'.
