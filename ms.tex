%% This is emulateapj reformatting of the AASTEX sample document
%%
%\documentclass[preprint2]{aastex}
\documentclass{emulateapj}

\usepackage{natbib}
\usepackage{amssymb,amsmath}
\usepackage{color}%,hyperref}
%\definecolor{linkcolor}{rgb}{0,0,0.5}
%\hypersetup{colorlinks=true,linkcolor=linkcolor,citecolor=linkcolor,
%            filecolor=linkcolor,urlcolor=linkcolor}
%\usepackage{url}

\usepackage[backref,breaklinks,colorlinks,citecolor=blue]{hyperref}
\usepackage[all]{hypcap}
\renewcommand*{\backref}[1]{[#1]}

\usepackage{acronym}


\bibliographystyle{apj}

\newcommand{\vdag}{(v)^\dagger}
\newcommand{\myemail}{tdm@astro.princeton.edu}

\newcommand{\ntotal}{8826}
\newcommand{\nfail}{1584}
\newcommand{\ncalc}{7242}
\newcommand{\nval}{1500}
\newcommand{\nvalnew}{1000}
\newcommand{\nfp}{500}
\newcommand{\nfpnew}{200}
\newcommand{\nbadphot}{1009} %EmptyPhotometry or MissingKOI
\newcommand{\nbadphotFP}{897}
\newcommand{\nbadphotCAND}{111}
\newcommand{\nbadrowemcmc}{210}
\newcommand{\nbadstellar}{80} %MissingStellar
\newcommand{\nbadsec}{82} %NoWeakSecondary
\newcommand{\nattempted}{7317}
\newcommand{\ntryfail}{75}
\newcommand{\kepler}{\textit{Kepler}}
\newcommand{\vespa}{\texttt{vespa}}
\newcommand{\isochrones}{\texttt{isochrones}}
\newcommand{\bvec}[1]{{\ensuremath{\boldsymbol{#1}}}}

%% TODO commands
\newcommand{\todo}[3]{{\color{#2} \emph{#1} TODO: #3}}
\newcommand{\tdmtodo}[1]{\todo{TDM}{red}{#1}}

\defcitealias{Morton:2012}{M12}
\defcitealias{Huber:2014}{H14}
\defcitealias{Dressing:2013}{D13}

%% You can insert a short comment on the title page using the command below.

\slugcomment{}

\shorttitle{False Positive Probabilities for KOIs}
\shortauthors{Morton et al.}

%% This is the end of the preamble.  Indicate the beginning of the
%% paper itself with \begin{document}.

\begin{document}

%% LaTeX will automatically break titles if they run longer than
%% one line. However, you may use \\ to force a line break if
%% you desire.

\title{False Positive Probabilities for all Kepler Objects of Interest: \\
        \nvalnew\ newly validated planets and \nfpnew\ likely false positives}

%% Use \author, \affil, and the \and command to format
%% author and affiliation information.
%% Note that \email has replaced the old \authoremail command
%% from AASTeX v4.0. You can use \email to mark an email address
%% anywhere in the paper, not just in the front matter.
%% As in the title, use \\ to force line breaks.

\author{Timothy D. Morton}
\affil{Department of Astrophysical Sciences, Princeton University}

%\altaffiltext{1}{Department of Astrophysics, Princeton University}

%% Mark off your abstract in the ``abstract'' environment. In the manuscript
%% style, abstract will output a Received/Accepted line after the
%% title and affiliation information. No date will appear since the author
%% does not have this information. The dates will be filled in by the
%% editorial office after submission.

\begin{abstract}
We present the results of applying a fully automated transit signal
false positive probability calculating procedure to every \ac{koi}.
Out of \ncalc\ \acp{koi}, we determine that \nval\ have probabilities
$<$1\% to be astrophysical false positives, and thus may be considered
validated planets.  \nvalnew\ of these have not yet been validated or
confirmed by other methods.  In addition, we identify \nfp\ KOIs
likely to be false positives ($>$90\% probability), \nfpnew\ of which
have not yet been identified as such. A side product of these
calculations is full stellar property posterior samplings for every
host star, modeled as single, binary, and triple systems.  These
calculations use \vespa, a publicly available Python package able to
be easily applied to any transiting exoplanet candidate.
\end{abstract}

%% Keywords should appear after the \end{abstract} command. The uncommented
%% example has been keyed in ApJ style. See the instructions to authors
%% for the journal to which you are submitting your paper to determine
%% what keyword punctuation is appropriate.

%% Authors who wish to have the most important objects in their paper
%% linked in the electronic edition to a data center may do so in the
%% subject header.  Objects should be in the appropriate "individual"
%% headers (e.g. quasars: individual, stars: individual, etc.) with the
%% additional provision that the total number of headers, including each
%% individual object, not exceed six.  The \objectname{} macro, and its
%% alias \object{}, is used to mark each object.  The macro takes the object
%% name as its primary argument.  This name will appear in the paper
%% and serve as the link's anchor in the electronic edition if the name
%% is recognized by the data centers.  The macro also takes an optional
%% argument in parentheses in cases where the data center identification
%% differs from what is to be printed in the paper.

\keywords{}


\section{Introduction}
\acresetall

The \kepler\ mission has revolutionized our understanding of
exoplanets.  Among many other important discoveries, \kepler\ has
identified several previously unsuspected features of planetary
systems, such as the prevalence of planets between the size of Earth
and Neptune, and a population of very compact multiple-planet
systems. And perhaps most notably, it has enabled for the first time
estimates of the occurrence rates of small planets ($\gtrsim$1
$R_\oplus$) out to orbits of about one year.  It is important to
remember, however, that these revolutionary discoveries depend
intimately on another revolution---how to interpret transiting planet
\textit{candidate} signals in the absence of unambiguous positive
confirmation of their veracity.

Before \kepler, every survey searching for transiting exoplanets had a 

\ac{fpp} and then \ac{fpp}.

%%%%%%%%%%%%%%%%%%%%%%%%%%%%%%%%%%%%%%%%%%%%%%%%%%%%%%

\section{Methods}
\label{sec:methods}

In this work, we apply the fully automated \ac{fpp}-computing
procedure described in \citet[][hereafter
  \citetalias{Morton:2012}]{Morton:2012} to \ncalc\ \acp{koi} (see
\autoref{sec:data} for details).  While we refer the reader to
\citetalias{Morton:2012} for a detailed description of the method, we
outline it briefly in this section.  The procedure as implemented here
is now publicly available in the Python module
\vespa\footnote{\url{https://github.com/timothydmorton/vespa}}
\citep{vespa}.

%%%%%%%%%%%%%%%%%%%%%%%%%%%%%%%

\subsection{False Positive Probabilities}
\label{sec:methods:fpp}

The basic idea of \vespa\ is to assign probabilities to different
astrophysical hypotheses that might describe a transiting planet
candidate signal.  If $\{H_i\}$ is the set of all considered
hypotheses, the probability for any given model $i$ is
\begin{equation}
  \label{eq:prob}
  \mathrm{Pr}\left(H_i\right) = \frac{\pi_i \mathcal
    L_i}{\displaystyle \sum_j \pi_j \mathcal L_j},
\end{equation}
where $\pi_i$ is the ``hypothesis prior'' and $\mathcal L_i$ is the
``hypothesis likelihood''\footnote{This factor is more widely known as
  the ``Bayesian evidence'' or ``marginalized likelihood'';
  \citet{Morton:2014b} argues for the term ``hypothesis likelihood,''
  as it can be clarifying to think of it that way.}
The prior represents how intrinsically probable the hypothesized
scenario is to exist, and the likelihood represents how closely the
shape of the observed transit signal matches with the expected shape
of a signal produced by the hypothesis.

\vespa\ models an eclipse signal as a simple trapezoid, parametrized
by depth $\delta$, total duration $T$, and shape parameter $T / \tau$,
where $\tau$ is the ``ingress/egress'' duration (such that a
completely V-shaped transit has $T/\tau = 2$).  For the transit signal
being evaluated, the joint posterior probability density function
(PDF) of these shape parameters is sampled with \ac{mcmc}, using the
\texttt{emcee} sampler \citep{emcee}.  This allows the likelihood for
each hypothesis to be determined by simulating a physically realistic
population of the hypothesized astrophysical scenario and using this
population to define the PDF for the trapezoidal parameters under the
hypothesis.  The likelihood is then
\begin{equation}
  \label{eq:lhood}
  \mathcal L_i = \displaystyle \int p_\mathrm{sig}\left(\bvec{\theta}\right)
                                    p_i\left(\bvec{\theta}\right)\,d\bvec{\theta},
\end{equation}
where $\bvec{\theta}$ is the vector of trapezoidal shape parameters,
$p_\mathrm{sig}$ is the posterior PDF of the signal, and $p_i$ is the
PDF for the parameters under hypothesis $i$.\footnote{$\mathcal L_i$
  may be seen to be the ``evidence'' or ``marginalized likelihood'' of
  the trapezoidal model under hypothesis $i$, with $p_\mathrm{sig}$
  being the likelihood and $p_i$ being the prior, integrated over the
  $\bvec{\theta}$ parameter space.  But for clarity, and for
  continuity with previous publication, we continue to call $\mathcal
  L_i$ the ``likelihood'' for hypothesis $i$.}

Observational constraints are incorporated in two different ways.
First, photometric (or spectroscopic/asteroseismic) measurements of
the target star are folded into the population simulations of each
hypothesis (see \autoref{sec:methods:stellar}).  All other constraints are
applied to narrow down which simulated instances of each scenario may
be counted in the final prior and likelihood evalulations; for example,
only blended eclipsing binaries with secondary eclipse depths
shallower than the observed limits contribute to the construction of
the $p_i$ trapezoidal shape parameter PDF.

The steps \vespa\ takes to calculate the \ac{fpp} of a transit signal
are thus as follows:
\begin{enumerate}
\item Fit trapezoid model to observed transit signal using \ac{mcmc}.
\item Generate population simulations for each hypothesis scenario
  being considered (conditioned on available observations of the
  target star; see \autoref{sec:methods:stellar}).
\item Fit each simulated eclipse in each scenario with a trapezoid
  model (using least-squares optimization).
\item Evaluate priors and likelihoods for each hypothesis, taking into
  account all available observational constraints, in order to
  calculate the \ac{fpp}.
\end{enumerate}

%%%%%%%%%%%%%%%%%%%%%%

\subsection{Stellar Properties}
\label{sec:methods:stellar}

The most substantial difference between the current implementation of
\vespa\ and the procedure documented in \citetalias{Morton:2012} is
how stellar properties are treated.  Previously, either the target
star's mass and radius were explicitly provided, or they were randomly
generated according to the stellar population expected along the line
of sight by the \ac{trilegal} Galactic stellar popultion synthesis
tool, but constrained to agree with some observed color(s) of the star
(e.g. $J-K$), to within some specified tolerance.  This strategy was
used both to generate the host stars for the transiting planet model
and the binary and triple stars for the EB and HEB false positive
models.

The new method now used by \vespa\ uses the \isochrones\ Python module
\citep{isochrones} to fold in observational constraints on the host
star.  At its core, \isochrones\ performs 3-D linear interpolation in
mass--[Fe/H]--age parameter space for a given stellar model grid.
This method of stellar modeling for FPP calculation debuted in
\citet{Montet:2015} and is explained there in more detail.  Instead of
randomly generating stars (or binary or triple systems of stars) from
a predefined distribution and culling them to agree with observed
colors, \emph{all} available constraints on the target star are used
to condition a direct fit of either a single--, binary--, or
triple--star model to the Dartmouth grid of stellar models
\citep{Dotter:2008, Feiden:2011}.  This fit is done using multi-modal
nested sampling, implemented with \texttt{MultiNest}
\citep{Feroz:2009, Feroz:2011, Feroz:2013}, via the
\texttt{PyMultiNest} wrapper \citep{Buchner:2014}.  Monte Carlo
samples of stellar properties for the population simulations are then
drawn directly from these posterior samples.

As a result, \vespa\ creates full posterior samplings of the physical
properties of the host star, modeled as a single, binary, and triple
star system, as a by-product of the \ac{fpp} calculation.  Parameters
directly fitted for in this process are stellar mass, age, [Fe/H],
$A_V$ extinction, and distance.  For binary and triple fits, secondary
and/or tertiary mass parameters are added, with all other parameters
assumed to be the same among all components.  Photometric observations
upon which these fits are conditioned are assumed to be the sum of all
components.  If spectroscopic and/or asteroseismic measurements are
used (e.g., constraints on effective temperature or stellar surface
gravity), they are assumed to relate to only the primary star.  Priors
used in these fits are listed in \autoref{table:priors}---notably, we
use a prior on [Fe/H] based on a double-Gaussian fit to the local
metallicity distribution \citep{Hayden:2015, Casagrande:2011}.  Posterior
chains of all other stellar parameters of interest (e.g., temperature,
surface gravity, radius, etc.) are derived from the chains of fitting
parameters by evaluting the stellar models using \isochrones.

\capstartfalse
\begin{deluxetable}{cc}
\tablewidth{0pt}
\tabletypesize{\scriptsize}
\tablecaption{Priors used in stellar property fits
\label{table:priors}}
\tablehead{
\colhead{Parameter} &
\colhead{Prior}}
\startdata
Primary mass $M_A$ & $\propto M_A^{-2.35},~M_A > 0.1$ \\
Secondary mass $M_B$ & $\propto (M_B/M_A)^{0.3},~0.1 <= M_B < M_A$ \\
Tertiary mass $M_C$ & $\propto (M_C/M_A)^{0.3},~0.1 <= M_C < M_B$ \\
Age {[}Gyr{]} & $\mathcal U(1,15)$ \tablenotemark{a}\\
{[}Fe/H{]} & $\frac{0.8}{0.15} \mathcal N(0.016, 0.15) + \frac{0.2}{0.22} \mathcal N(-0.15, 0.22)$  \tablenotemark{b} \\
$A_V$ {[}mag{]} & $\mathcal U(0, A_{V, \mathrm{max}})$ \tablenotemark{c} \\
Distance $d$ & $\propto d^2$ 
\enddata
\tablenotetext{a}{The age range for the Dartmouth stellar model grids used.}
\tablenotetext{b}{Double-Gaussian fit to measured local stellar
  metallicity distribution \citep{Hayden:2015, Casagrande:2011}.}
\tablenotetext{3}{Maximum allowed value is the Galactic extinction at
  infinity calculated along the star's line of sight, according to \citet{Schlegel:1998}.}
\end{deluxetable}
\capstarttrue

%%%%%%%%%%%%%%%%%%%%%%%%%%%%%%%%%%%%%%%%%%%%%%%%%%%%%%

\section{Data and Constraints}
\label{sec:data}

The goal of this work is to calculate the \ac{fpp} for every \ac{koi},
regardless of classification as CONFIRMED, CANDIDATE, or FALSE
POSITIVE.  As such, we begin with a list of \ntotal\ \acp{koi} from
the cumulative table at the \ac{nexsci} Exoplanet Archive.  We then
gather ancillary data and constraints from various sources in order to
enable the \vespa\ calculation:

\begin{enumerate}
\item The RA/Dec coordinates of each star from the \ac{kic}.
\item $grizJHK$ photometry from the \ac{kic}, with $griz$ bands
  corrected to the \ac{sdss} photometric scale according to
  \citet{Pinsonneault:2012}.
\item Stellar $T_\mathrm{eff}$, [Fe/H], and $\log g$ values and
  uncertainties from the \citet{Huber:2014} stellar properties
  catalog, if the provenance of these values is from spectroscopy or
  asteroseismology.
\item Detrended \kepler\ photometry used for the \ac{mcmc} modeling of
  \citet{Rowe:2015}, along with information about individually fitted
  transit times, where available.  
\item Best-fit $R_p/R_\star$ from the \citet{Rowe:2015} MCMC analysis.
\item Centroid uncertainty information from the \ac{nexsci} Exoplanet
  Archive: we assume that the allowed ``exclusion'' radius for a blend
  scenario is 3$\times$ the uncertainty in the fitted centroid
  position (the \verb|koi_dicco_msky_err| column in the Archive
  table).  We floor this value at 0\farcs5, to prevent unrealistically
  small exclusion radii.  If this quantity is not available from the
  Archive we set a default exclusion radius of 4\arcsec.
\item The maximum secondary eclipse depth allowed by the
  \kepler\ photometry.  This quantity is obtained from vetting
  metrics produced internally by the \kepler\ Science Office.  When
  these metrics are not available for a particular \ac{koi}, we
  default to 10$\times$ the uncertainty in the \kepler\ pipeline
  measured transit depth (\verb|koi_depth_err1|).  While these
  secondary metrics are not yet available on the Archive, they should
  become available with the final data release.  
  
\end{enumerate}

%This leaves \nattempted\ \acp{koi} upon which we run \vespa.
%\ntryfail of these encounter runtime errors during the
%\vespa\ calculation; these are further explained below in
%\autoref{sec:failures}.


%%%%%%%%%%%%%%%%%%%%%%%%%%%%%%%%%%%%%%%%%%%%%%%%%%%%%%

\section{Results}
\label{sec:results}

The results of the \vespa\ calculations are presented in
\autoref{table:stars} and \autoref{table:fpp}, and are discussed in
the following subsections.


%%%%%%%%%%%%%%%%%%%%%%%%%%%%%%%

\subsection{Stellar Properties}
\label{sec:results:stars}

As discussed in \autoref{sec:methods:stellar}, \vespa\ fits for stellar
properties as part of its \ac{fpp}--calculating procedure, using the
\isochrones\ package.  Thus, we obtain posterior samplings of the
physical properties of each \ac{koi} as a side effect of this batch
calculation, a result of general interest independent of \acp{fpp}.
\autoref{table:stars} presents summarized results of the single--star
fits.  While \vespa\ also fits double-- and triple--star models for
each \ac{koi}, these are of less general interest and so we do not
present them separately.

\autoref{fig:starsteff} and \autoref{fig:starsfehradius} compare the
estimated effective temperatures, metallicities, and radii, derived
here to those independently determined for the official
\kepler\ stellar properties catalog \citep[][hereafter
  \citetalias{Huber:2014}]{Huber:2014}.  While there is largely
general agreement, there are also discrepancies, highlighting the
difficulty of estimating physical stellar properties.

In particular, we note that for stars which \citetalias{Huber:2014}
list as $T_{\rm eff} < 4000$\,K, \isochrones\ predicts systematically
hotter temperatures.  Most of the \citetalias{Huber:2014} properties
for these stars are taken from
\citet[][\citetalias{Dressing:2013}]{Dressing:2013}.  Those properties
were determined by trying to match the colors of each star to the colors
of a grid of model stars from the Dartmouth models, supplemented by
some interpolation.  \citetalias{Dressing:2013} also imposed priors
on [Fe/H] and the height of stars above the plane of the Galaxy.  To
validate their methodology, \citetalias{Dressing:2013} compare their
results for 26 nearby stars to the masses predicted for those stars by
combining parallax measurements with the \citet{Delfosse:2000}
relation between mass and absolute $K$-band magnitude.  While they
find general good agreement, the \citetalias{Dressing:2013} masses are
noted to be on average about 5\% lower than the Delfosse-predicted masses. 



\begin{figure*}[p]
\begin{center}
\includegraphics[width=7in]{figures/hubercompare_teff.pdf}
\end{center}
\caption{Comparison between effective temperatures estimated from the
  \isochrones\ analysis in this work and those from the
  \kepler\ stellar parameters catalog \citep[][hereafter
    \citetalias{Huber:2014}]{Huber:2014}.  Bottom panel shows stars
  for which \citetalias{Huber:2014} predicts $T_{\rm eff} < 4500$\,K,
  middle spans $4500\,{\rm K} < T_{\rm eff} < 6500\,{\rm K}$, and top
  has $T_{\rm eff} > 6500$\,K. Blue horizontal bold lines are the
  \citetalias{Huber:2014} values in sorted order; blue shading
  represents the error bars from \citetalias{Huber:2014}.  Vertical
  lines span the 1$\sigma$ credible region of the \isochrones\ fits;
  these lines are grey if they overlap with the
  \citetalias{Huber:2014} 1$\sigma$ region and red (with the median
  marked by a point) if they are inconsistent.  This comparison shows
  that the stellar parameters estimated in this work are broadly
  consistent with \citetalias{Huber:2014}, though less so for the
  coolest and hottest stars.
\label{fig:starsteff}}
\end{figure*}


\begin{figure*}[p]
\begin{center}
\includegraphics[width=7in]{figures/hubercompare_fehradius.pdf}
\end{center}
\caption{Comparison between metallicities and radii estimated from the
  \isochrones\ analysis in this work and those from the
  \kepler\ stellar parameters catalog \citep[][hereafter
    \citetalias{Huber:2014}]{Huber:2014}.  Top panel shows metallicity
  for all stars in the sample.  The middle panel shows stars for which
  \citetalias{Huber:2014} estimates $R_\star < 2\,R_\odot$, and the
  bottom shows $R_\star > 2\,R_\odot$. Blue horizontal bold lines are
  the \citetalias{Huber:2014} values in sorted order; blue shading
  represents the error bars from \citetalias{Huber:2014}.  Vertical
  lines span the 1$\sigma$ credible region of the \isochrones\ fits;
  these lines are grey if they overlap with the
  \citetalias{Huber:2014} 1$\sigma$ region and red (with the median
  marked by a point) if they are inconsistent.  This comparison shows
  that the stellar parameters estimated in this work are broadly
  consistent with \citetalias{Huber:2014}, though less so for the more
  evolved stars.  The metallicity estimates of the
  \isochrones\ calculations are driven by the use of the local
  metallicity prior \citep[][\autoref{table:priors}]{Hayden:2015,
    Casagrande:2011}.
\label{fig:starsfehradius}}
\end{figure*}



%%%%%%%%%%%%%%%%%%%%%%%%%%%%%%%

\subsection{False Positive Probabilities}
\label{sec:results:fpp}

Of the \ntotal\ \acp{koi} in the cumulative
table at the \ac{nexsci} Exoplanet Archive, \vespa\ successfully
calculated the \ac{fpp} for \ncalc.  Most of the \nfail\ failures are a
result of a \ac{koi} no longer being ``of interest,'' despite still
being listed in the cumulative table.  \autoref{sec:failures} contains
detailed explanations of the different failure modes.



%%%%%%%%%%%%%%%%%%%%%%%%%%%%%%%%%%%%%%%%%%%%%%%%%%%%%%

\subsection{Failure Modes}
\label{sec:failures}


Of the \ntotal\ total \acp{koi}, \nbadphot\ did not receive \ac{mcmc}
modeling.  Most of these (\nbadphotFP) are already designated FALSE
POSITIVE at the archive, though \nbadphotCAND\ have CANDIDATE
disposition.  One, K00245.04, is a confirmed planet.  \tdmtodo{Find
  out why??}  An additional \nbadrowemcmc\ did receive \ac{mcmc}
modeling but had unphysical fit results; e.g., negative $R_p/R_\star$
or best-fit impact parameter greater than $(1 + R_p/R_\star)$.  Of the
\acp{koi} that did have usable \ac{mcmc} fits, \nbadsec\ had no
\verb|koi_depth_err1| value on the archive, and thus had no weak
secondary constraint and were left out of the calculations.


%%%%%%%%%%%%%%%%%%%%%%%%%%%%%%%%%%%%%%%%%%%%%%%%%%%%%%

\section{Conclusions}
\label{sec:conclusions}

\acrodef{fpp}[FPP]{false positive probability}
\acrodef{koi}[KOI]{\kepler\ Object of Interest}
\acrodefplural{koi}[KOIs]{\kepler\ Objects of Interest}
\acrodef{nexsci}[NExScI]{NASA Exoplanet Science Institute}
\acrodef{mcmc}[MCMC]{Markov Chain Monte Carlo}
\acrodef{trilegal}[TRILEGAL]{TRIdimensional modeL of thE GALaxy}
\acrodef{kic}[KIC]{Kepler Input Catalog}
\acrodef{sdss}[SDSS]{Sloan Digital Sky Survey}
\acrodef{ttv}[TTV]{transit timing variations}


\acknowledgments
It is a pleasure to thank
\ldots\

\clearpage
\bibliography{ms}
\clearpage

\capstartfalse

\begin{deluxetable*}{lccccccccccc}
\tablewidth{0pt}
\tabletypesize{\scriptsize}
\tablecaption{Stellar Properties
\tablabel{stars}}
\tablehead{\colhead{KOI} &
    \colhead{$M_\star$} &
    \colhead{$R_\star$} &
    \colhead{$T_{\rm eff}$} &
    \colhead{$\log g$} &
    \colhead{[Fe/H]} &
    \colhead{Age} &
    \colhead{$d$} &
    \colhead{$A_V$}&
    \colhead{$\pi\left(T_{\rm eff}\right)$} &
    \colhead{$\pi\left(\log g\right)$} &
    \colhead{$\pi\left({\rm [Fe/H]}\right)$} \\
    \colhead{} &
    \colhead{($M_\odot$)} &
    \colhead{($R_\odot$)} &
    \colhead{(K)} &
    \colhead{(cgs)} &
    \colhead{(dex)} &
    \colhead{(Gyr)} &
    \colhead{(pc)} &
    \colhead{(mag)} &
    \colhead{} &
    \colhead{} &
    \colhead{} 
    }
\startdata
K00757.03 &${ 0.80 }^{ +0.04 }_{ -0.04 }$&${ 0.76 }^{ +0.03 }_{ -0.04 }$&${ 5017 }^{ +54 }_{ -57 }$&${ 4.59 }^{ +0.03 }_{ -0.05 }$&${ -0.04 }^{ +0.15 }_{ -0.17 }$&${ 9.69 }^{ +0.37 }_{ -0.41 }$&${ 879 }^{ +42 }_{ -49 }$&${ 0.10 }^{ +0.06 }_{ -0.06 }$&--&--&--\\ 
K00758.01 &${ 0.79 }^{ +0.04 }_{ -0.04 }$&${ 0.74 }^{ +0.03 }_{ -0.04 }$&${ 4913 }^{ +94 }_{ -87 }$&${ 4.60 }^{ +0.02 }_{ -0.04 }$&${ -0.03 }^{ +0.15 }_{ -0.17 }$&${ 9.67 }^{ +0.36 }_{ -0.38 }$&${ 651 }^{ +32 }_{ -37 }$&${ 0.17 }^{ +0.12 }_{ -0.11 }$&--&--&--\\ 
K00759.02 &${ 0.84 }^{ +0.05 }_{ -0.05 }$&${ 0.81 }^{ +0.06 }_{ -0.06 }$&${ 5328 }^{ +60 }_{ -55 }$&${ 4.56 }^{ +0.04 }_{ -0.08 }$&${ -0.14 }^{ +0.16 }_{ -0.19 }$&${ 9.64 }^{ +0.39 }_{ -0.43 }$&${ 775 }^{ +64 }_{ -61 }$&${ 0.06 }^{ +0.05 }_{ -0.04 }$&--&--&--\\ 
K00760.01 &${ 1.06 }^{ +0.07 }_{ -0.06 }$&${ 1.07 }^{ +0.25 }_{ -0.10 }$&${ 5895 }^{ +86 }_{ -102 }$&${ 4.40 }^{ +0.08 }_{ -0.17 }$&${ 0.06 }^{ +0.15 }_{ -0.15 }$&${ 9.59 }^{ +0.20 }_{ -0.35 }$&${ 1356 }^{ +311 }_{ -141 }$&${ 0.14 }^{ +0.05 }_{ -0.08 }$&--&--&--\\ 
K00761.01 &${ 0.94 }^{ +0.04 }_{ -0.04 }$&${ 0.92 }^{ +0.19 }_{ -0.06 }$&${ 5505 }^{ +69 }_{ -90 }$&${ 4.49 }^{ +0.05 }_{ -0.17 }$&${ 0.09 }^{ +0.13 }_{ -0.13 }$&${ 9.71 }^{ +0.35 }_{ -0.46 }$&${ 1051 }^{ +223 }_{ -75 }$&${ 0.32 }^{ +0.04 }_{ -0.09 }$&--&--&--\\ 
K00762.01 &${ 1.03 }^{ +0.09 }_{ -0.08 }$&${ 1.05 }^{ +0.25 }_{ -0.12 }$&${ 5887 }^{ +165 }_{ -179 }$&${ 4.41 }^{ +0.09 }_{ -0.17 }$&${ 0.01 }^{ +0.14 }_{ -0.17 }$&${ 9.58 }^{ +0.24 }_{ -0.34 }$&${ 1330 }^{ +317 }_{ -168 }$&${ 0.28 }^{ +0.12 }_{ -0.15 }$&--&--&--\\ 
K00763.01 &${ 0.99 }^{ +0.04 }_{ -0.05 }$&${ 0.98 }^{ +0.18 }_{ -0.08 }$&${ 5710 }^{ +73 }_{ -91 }$&${ 4.45 }^{ +0.06 }_{ -0.15 }$&${ 0.05 }^{ +0.14 }_{ -0.17 }$&${ 9.67 }^{ +0.27 }_{ -0.39 }$&${ 1304 }^{ +241 }_{ -109 }$&${ 0.15 }^{ +0.04 }_{ -0.08 }$&--&--&--\\ 
K00764.01 &${ 0.91 }^{ +0.04 }_{ -0.04 }$&${ 0.87 }^{ +0.07 }_{ -0.05 }$&${ 5327 }^{ +53 }_{ -70 }$&${ 4.53 }^{ +0.04 }_{ -0.09 }$&${ 0.15 }^{ +0.15 }_{ -0.16 }$&${ 9.67 }^{ +0.37 }_{ -0.44 }$&${ 907 }^{ +76 }_{ -52 }$&${ 0.21 }^{ +0.03 }_{ -0.07 }$&--&--&--\\ 
K00765.01 &${ 0.87 }^{ +0.04 }_{ -0.05 }$&${ 0.82 }^{ +0.06 }_{ -0.05 }$&${ 5352 }^{ +61 }_{ -61 }$&${ 4.55 }^{ +0.03 }_{ -0.07 }$&${ -0.06 }^{ +0.15 }_{ -0.17 }$&${ 9.59 }^{ +0.40 }_{ -0.40 }$&${ 877 }^{ +70 }_{ -59 }$&${ 0.09 }^{ +0.05 }_{ -0.06 }$&--&--&--\\ 
K00766.01 &${ 1.15 }^{ +0.10 }_{ -0.07 }$&${ 1.20 }^{ +0.32 }_{ -0.13 }$&${ 6038 }^{ +98 }_{ -109 }$&${ 4.34 }^{ +0.08 }_{ -0.18 }$&${ 0.14 }^{ +0.15 }_{ -0.14 }$&${ 9.50 }^{ +0.15 }_{ -0.29 }$&${ 1778 }^{ +463 }_{ -208 }$&${ 0.14 }^{ +0.05 }_{ -0.08 }$&--&--&--\\ 
K00767.01 &${ 0.99 }^{ +0.05 }_{ -0.06 }$&${ 0.99 }^{ +0.22 }_{ -0.09 }$&${ 5709 }^{ +89 }_{ -134 }$&${ 4.44 }^{ +0.07 }_{ -0.17 }$&${ 0.07 }^{ +0.14 }_{ -0.15 }$&${ 9.68 }^{ +0.25 }_{ -0.41 }$&${ 980 }^{ +218 }_{ -98 }$&${ 0.36 }^{ +0.06 }_{ -0.12 }$&--&--&--\\ 
K00768.01 &${ 0.81 }^{ +0.03 }_{ -0.04 }$&${ 0.77 }^{ +0.03 }_{ -0.03 }$&${ 5035 }^{ +47 }_{ -49 }$&${ 4.58 }^{ +0.02 }_{ -0.05 }$&${ 0.00 }^{ +0.13 }_{ -0.15 }$&${ 9.68 }^{ +0.36 }_{ -0.40 }$&${ 857 }^{ +38 }_{ -39 }$&${ 0.08 }^{ +0.05 }_{ -0.06 }$&--&--&--\\ 
K00769.01 &${ 0.97 }^{ +0.05 }_{ -0.04 }$&${ 0.95 }^{ +0.17 }_{ -0.06 }$&${ 5624 }^{ +75 }_{ -91 }$&${ 4.47 }^{ +0.05 }_{ -0.15 }$&${ 0.08 }^{ +0.15 }_{ -0.14 }$&${ 9.65 }^{ +0.32 }_{ -0.43 }$&${ 1139 }^{ +200 }_{ -87 }$&${ 0.15 }^{ +0.05 }_{ -0.08 }$&--&--&--\\ 
K00770.01 &${ 0.93 }^{ +0.04 }_{ -0.05 }$&${ 0.89 }^{ +0.10 }_{ -0.06 }$&${ 5565 }^{ +88 }_{ -106 }$&${ 4.51 }^{ +0.04 }_{ -0.10 }$&${ -0.02 }^{ +0.15 }_{ -0.17 }$&${ 9.63 }^{ +0.33 }_{ -0.41 }$&${ 1114 }^{ +127 }_{ -77 }$&${ 0.16 }^{ +0.07 }_{ -0.10 }$&--&--&--\\ 
K00772.01 &${ 1.05 }^{ +0.06 }_{ -0.06 }$&${ 1.05 }^{ +0.21 }_{ -0.09 }$&${ 5917 }^{ +100 }_{ -85 }$&${ 4.41 }^{ +0.07 }_{ -0.14 }$&${ 0.01 }^{ +0.13 }_{ -0.17 }$&${ 9.56 }^{ +0.22 }_{ -0.32 }$&${ 1369 }^{ +274 }_{ -127 }$&${ 0.09 }^{ +0.07 }_{ -0.06 }$&--&--&--\\ 
K00773.01 &${ 0.95 }^{ +0.04 }_{ -0.05 }$&${ 0.92 }^{ +0.11 }_{ -0.06 }$&${ 5627 }^{ +75 }_{ -80 }$&${ 4.50 }^{ +0.05 }_{ -0.11 }$&${ -0.01 }^{ +0.15 }_{ -0.18 }$&${ 9.61 }^{ +0.33 }_{ -0.39 }$&${ 1011 }^{ +128 }_{ -74 }$&${ 0.12 }^{ +0.05 }_{ -0.07 }$&--&--&--\\ 
K00774.01 &${ 0.97 }^{ +0.06 }_{ -0.07 }$&${ 0.96 }^{ +0.16 }_{ -0.09 }$&${ 5760 }^{ +109 }_{ -104 }$&${ 4.46 }^{ +0.07 }_{ -0.13 }$&${ -0.07 }^{ +0.15 }_{ -0.21 }$&${ 9.65 }^{ +0.27 }_{ -0.40 }$&${ 1170 }^{ +189 }_{ -113 }$&${ 0.12 }^{ +0.08 }_{ -0.08 }$&--&--&--\\ 
K00775.01 &${ 0.66 }^{ +0.02 }_{ -0.03 }$&${ 0.63 }^{ +0.02 }_{ -0.02 }$&${ 4242 }^{ +42 }_{ -36 }$&${ 4.66 }^{ +0.01 }_{ -0.02 }$&${ -0.00 }^{ +0.12 }_{ -0.09 }$&${ 9.55 }^{ +0.36 }_{ -0.35 }$&${ 336 }^{ +12 }_{ -15 }$&${ 0.05 }^{ +0.06 }_{ -0.03 }$&(4117, 92)&(4.66, 0.03)&(0.04, 0.14)\\ 
K00775.02 &${ 0.66 }^{ +0.02 }_{ -0.03 }$&${ 0.63 }^{ +0.02 }_{ -0.02 }$&${ 4241 }^{ +42 }_{ -37 }$&${ 4.66 }^{ +0.01 }_{ -0.02 }$&${ -0.00 }^{ +0.11 }_{ -0.09 }$&${ 9.54 }^{ +0.37 }_{ -0.32 }$&${ 336 }^{ +12 }_{ -15 }$&${ 0.04 }^{ +0.06 }_{ -0.03 }$&(4117, 92)&(4.66, 0.03)&(0.04, 0.14)\\ 
K00775.03 &${ 0.66 }^{ +0.02 }_{ -0.03 }$&${ 0.63 }^{ +0.02 }_{ -0.02 }$&${ 4241 }^{ +40 }_{ -34 }$&${ 4.66 }^{ +0.01 }_{ -0.02 }$&${ -0.01 }^{ +0.11 }_{ -0.09 }$&${ 9.53 }^{ +0.37 }_{ -0.32 }$&${ 336 }^{ +11 }_{ -14 }$&${ 0.04 }^{ +0.06 }_{ -0.03 }$&(4117, 92)&(4.66, 0.03)&(0.04, 0.14)\\ 
K00776.01 &${ 0.88 }^{ +0.04 }_{ -0.05 }$&${ 0.84 }^{ +0.07 }_{ -0.05 }$&${ 5355 }^{ +60 }_{ -72 }$&${ 4.54 }^{ +0.04 }_{ -0.08 }$&${ 0.00 }^{ +0.15 }_{ -0.17 }$&${ 9.65 }^{ +0.38 }_{ -0.42 }$&${ 978 }^{ +80 }_{ -65 }$&${ 0.14 }^{ +0.04 }_{ -0.08 }$&--&--&--\\ 
K00777.01 &${ 0.82 }^{ +0.04 }_{ -0.05 }$&${ 0.79 }^{ +0.05 }_{ -0.05 }$&${ 5195 }^{ +56 }_{ -71 }$&${ 4.57 }^{ +0.04 }_{ -0.07 }$&${ -0.08 }^{ +0.16 }_{ -0.20 }$&${ 9.69 }^{ +0.36 }_{ -0.41 }$&${ 817 }^{ +57 }_{ -60 }$&${ 0.17 }^{ +0.04 }_{ -0.08 }$&--&--&--\\ 
K00778.01 &${ 0.60 }^{ +0.04 }_{ -0.03 }$&${ 0.57 }^{ +0.03 }_{ -0.02 }$&${ 4192 }^{ +41 }_{ -43 }$&${ 4.69 }^{ +0.01 }_{ -0.02 }$&${ -0.27 }^{ +0.14 }_{ -0.14 }$&${ 9.52 }^{ +0.42 }_{ -0.33 }$&${ 302 }^{ +19 }_{ -18 }$&${ 0.06 }^{ +0.05 }_{ -0.04 }$&(4128, 100)&(4.70, 0.02)&(-0.36, 0.16)\\ 
K00779.01 &${ 0.97 }^{ +0.04 }_{ -0.05 }$&${ 0.95 }^{ +0.15 }_{ -0.07 }$&${ 5652 }^{ +83 }_{ -102 }$&${ 4.47 }^{ +0.06 }_{ -0.14 }$&${ 0.03 }^{ +0.13 }_{ -0.16 }$&${ 9.67 }^{ +0.29 }_{ -0.42 }$&${ 1233 }^{ +200 }_{ -100 }$&${ 0.20 }^{ +0.05 }_{ -0.09 }$&--&--&--\\ 
K00780.01 &${ 0.79 }^{ +0.04 }_{ -0.05 }$&${ 0.75 }^{ +0.03 }_{ -0.04 }$&${ 4945 }^{ +70 }_{ -71 }$&${ 4.60 }^{ +0.03 }_{ -0.04 }$&${ -0.05 }^{ +0.15 }_{ -0.18 }$&${ 9.67 }^{ +0.36 }_{ -0.39 }$&${ 651 }^{ +32 }_{ -38 }$&${ 0.14 }^{ +0.09 }_{ -0.09 }$&--&--&--\\ 
K00780.02 &${ 0.79 }^{ +0.04 }_{ -0.04 }$&${ 0.75 }^{ +0.03 }_{ -0.04 }$&${ 4944 }^{ +74 }_{ -71 }$&${ 4.60 }^{ +0.02 }_{ -0.04 }$&${ -0.05 }^{ +0.15 }_{ -0.17 }$&${ 9.66 }^{ +0.37 }_{ -0.39 }$&${ 651 }^{ +30 }_{ -36 }$&${ 0.14 }^{ +0.09 }_{ -0.09 }$&--&--&--\\ 
K00781.01 &${ 0.51 }^{ +0.02 }_{ -0.03 }$&${ 0.49 }^{ +0.02 }_{ -0.03 }$&${ 3701 }^{ +39 }_{ -40 }$&${ 4.77 }^{ +0.02 }_{ -0.02 }$&${ -0.02 }^{ +0.09 }_{ -0.09 }$&${ 9.63 }^{ +0.38 }_{ -0.39 }$&${ 243 }^{ +15 }_{ -17 }$&${ 0.14 }^{ +0.08 }_{ -0.08 }$&(3648, 65)&(4.79, 0.06)&(0.00, 0.14)\\ 
K00782.01 &${ 1.00 }^{ +0.05 }_{ -0.05 }$&${ 0.99 }^{ +0.21 }_{ -0.08 }$&${ 5723 }^{ +69 }_{ -88 }$&${ 4.45 }^{ +0.06 }_{ -0.16 }$&${ 0.06 }^{ +0.14 }_{ -0.16 }$&${ 9.66 }^{ +0.26 }_{ -0.41 }$&${ 1180 }^{ +253 }_{ -99 }$&${ 0.17 }^{ +0.04 }_{ -0.07 }$&--&--&--\\ 
K00783.01 &${ 0.94 }^{ +0.04 }_{ -0.05 }$&${ 0.91 }^{ +0.12 }_{ -0.06 }$&${ 5520 }^{ +73 }_{ -108 }$&${ 4.50 }^{ +0.05 }_{ -0.12 }$&${ 0.07 }^{ +0.16 }_{ -0.15 }$&${ 9.65 }^{ +0.35 }_{ -0.43 }$&${ 875 }^{ +113 }_{ -61 }$&${ 0.27 }^{ +0.05 }_{ -0.10 }$&--&--&--\\ 
K00784.01 &${ 0.59 }^{ +0.03 }_{ -0.04 }$&${ 0.57 }^{ +0.03 }_{ -0.03 }$&${ 4147 }^{ +45 }_{ -45 }$&${ 4.70 }^{ +0.02 }_{ -0.02 }$&${ -0.26 }^{ +0.13 }_{ -0.14 }$&${ 9.54 }^{ +0.38 }_{ -0.33 }$&${ 325 }^{ +19 }_{ -21 }$&${ 0.04 }^{ +0.06 }_{ -0.03 }$&(4059, 93)&(4.70, 0.02)&(-0.26, 0.18)\\ 
K00784.02 &${ 0.59 }^{ +0.03 }_{ -0.04 }$&${ 0.57 }^{ +0.03 }_{ -0.03 }$&${ 4147 }^{ +47 }_{ -46 }$&${ 4.70 }^{ +0.02 }_{ -0.02 }$&${ -0.26 }^{ +0.12 }_{ -0.14 }$&${ 9.53 }^{ +0.39 }_{ -0.32 }$&${ 325 }^{ +20 }_{ -22 }$&${ 0.05 }^{ +0.06 }_{ -0.03 }$&(4059, 93)&(4.70, 0.02)&(-0.26, 0.18)\\ 
K00785.01 &${ 0.87 }^{ +0.05 }_{ -0.05 }$&${ 0.84 }^{ +0.07 }_{ -0.06 }$&${ 5403 }^{ +97 }_{ -106 }$&${ 4.54 }^{ +0.04 }_{ -0.09 }$&${ -0.09 }^{ +0.16 }_{ -0.19 }$&${ 9.65 }^{ +0.38 }_{ -0.45 }$&${ 968 }^{ +90 }_{ -76 }$&${ 0.17 }^{ +0.09 }_{ -0.11 }$&--&--&--\\ 
K00786.01 &${ 1.08 }^{ +0.10 }_{ -0.06 }$&${ 1.11 }^{ +0.38 }_{ -0.11 }$&${ 5890 }^{ +77 }_{ -99 }$&${ 4.38 }^{ +0.08 }_{ -0.22 }$&${ 0.13 }^{ +0.15 }_{ -0.15 }$&${ 9.57 }^{ +0.18 }_{ -0.34 }$&${ 1342 }^{ +446 }_{ -145 }$&${ 0.21 }^{ +0.04 }_{ -0.08 }$&--&--&--\\ 
K00787.01 &${ 0.97 }^{ +0.06 }_{ -0.05 }$&${ 0.95 }^{ +0.20 }_{ -0.07 }$&${ 5654 }^{ +118 }_{ -108 }$&${ 4.47 }^{ +0.06 }_{ -0.16 }$&${ 0.03 }^{ +0.13 }_{ -0.16 }$&${ 9.65 }^{ +0.28 }_{ -0.40 }$&${ 1155 }^{ +238 }_{ -99 }$&${ 0.15 }^{ +0.10 }_{ -0.10 }$&--&--&--\\ 
K00787.02 &${ 0.96 }^{ +0.05 }_{ -0.05 }$&${ 0.94 }^{ +0.15 }_{ -0.07 }$&${ 5647 }^{ +123 }_{ -112 }$&${ 4.48 }^{ +0.05 }_{ -0.13 }$&${ 0.02 }^{ +0.13 }_{ -0.17 }$&${ 9.61 }^{ +0.32 }_{ -0.39 }$&${ 1142 }^{ +181 }_{ -93 }$&${ 0.15 }^{ +0.10 }_{ -0.10 }$&--&--&--\\ 
K00788.01 &${ 0.80 }^{ +0.04 }_{ -0.05 }$&${ 0.76 }^{ +0.03 }_{ -0.04 }$&${ 5021 }^{ +64 }_{ -62 }$&${ 4.58 }^{ +0.03 }_{ -0.04 }$&${ -0.04 }^{ +0.15 }_{ -0.16 }$&${ 9.70 }^{ +0.33 }_{ -0.39 }$&${ 666 }^{ +33 }_{ -37 }$&${ 0.11 }^{ +0.07 }_{ -0.08 }$&--&--&--\\ 
K00790.01 &${ 0.84 }^{ +0.04 }_{ -0.05 }$&${ 0.80 }^{ +0.06 }_{ -0.05 }$&${ 5261 }^{ +93 }_{ -81 }$&${ 4.56 }^{ +0.04 }_{ -0.07 }$&${ -0.09 }^{ +0.15 }_{ -0.18 }$&${ 9.67 }^{ +0.37 }_{ -0.44 }$&${ 825 }^{ +60 }_{ -60 }$&${ 0.12 }^{ +0.09 }_{ -0.08 }$&--&--&--\\ 
K00790.02 &${ 0.84 }^{ +0.04 }_{ -0.05 }$&${ 0.80 }^{ +0.05 }_{ -0.05 }$&${ 5257 }^{ +93 }_{ -74 }$&${ 4.56 }^{ +0.04 }_{ -0.07 }$&${ -0.09 }^{ +0.15 }_{ -0.18 }$&${ 9.65 }^{ +0.38 }_{ -0.42 }$&${ 823 }^{ +59 }_{ -58 }$&${ 0.11 }^{ +0.10 }_{ -0.08 }$&--&--&--\\ 
K00791.01 &${ 0.92 }^{ +0.04 }_{ -0.05 }$&${ 0.89 }^{ +0.09 }_{ -0.06 }$&${ 5559 }^{ +85 }_{ -82 }$&${ 4.51 }^{ +0.04 }_{ -0.10 }$&${ -0.05 }^{ +0.14 }_{ -0.17 }$&${ 9.60 }^{ +0.36 }_{ -0.39 }$&${ 951 }^{ +101 }_{ -68 }$&${ 0.10 }^{ +0.07 }_{ -0.07 }$&--&--&--\\ 
K00792.01 &${ 1.02 }^{ +0.09 }_{ -0.07 }$&${ 1.04 }^{ +0.21 }_{ -0.11 }$&${ 5908 }^{ +172 }_{ -172 }$&${ 4.42 }^{ +0.08 }_{ -0.14 }$&${ -0.03 }^{ +0.14 }_{ -0.18 }$&${ 9.58 }^{ +0.21 }_{ -0.32 }$&${ 1190 }^{ +245 }_{ -137 }$&${ 0.22 }^{ +0.13 }_{ -0.14 }$&--&--&--\\ 
K00793.01 &${ 1.12 }^{ +0.11 }_{ -0.09 }$&${ 1.17 }^{ +0.37 }_{ -0.15 }$&${ 6033 }^{ +119 }_{ -187 }$&${ 4.35 }^{ +0.10 }_{ -0.20 }$&${ 0.09 }^{ +0.14 }_{ -0.14 }$&${ 9.53 }^{ +0.18 }_{ -0.29 }$&${ 1257 }^{ +396 }_{ -168 }$&${ 0.46 }^{ +0.07 }_{ -0.14 }$&--&--&--\\ 
K00794.01 &${ 0.96 }^{ +0.06 }_{ -0.06 }$&${ 0.95 }^{ +0.16 }_{ -0.08 }$&${ 5703 }^{ +141 }_{ -149 }$&${ 4.47 }^{ +0.06 }_{ -0.13 }$&${ -0.03 }^{ +0.14 }_{ -0.18 }$&${ 9.64 }^{ +0.29 }_{ -0.38 }$&${ 976 }^{ +167 }_{ -94 }$&${ 0.23 }^{ +0.11 }_{ -0.13 }$&--&--&--\\ 
K00795.01 &${ 0.89 }^{ +0.06 }_{ -0.06 }$&${ 0.86 }^{ +0.08 }_{ -0.07 }$&${ 5493 }^{ +124 }_{ -119 }$&${ 4.53 }^{ +0.05 }_{ -0.09 }$&${ -0.11 }^{ +0.16 }_{ -0.19 }$&${ 9.64 }^{ +0.36 }_{ -0.43 }$&${ 1066 }^{ +98 }_{ -93 }$&${ 0.18 }^{ +0.11 }_{ -0.11 }$&--&--&--\\ 
K00796.01 &${ 0.80 }^{ +0.05 }_{ -0.06 }$&${ 0.76 }^{ +0.05 }_{ -0.06 }$&${ 5132 }^{ +107 }_{ -87 }$&${ 4.58 }^{ +0.04 }_{ -0.05 }$&${ -0.15 }^{ +0.16 }_{ -0.21 }$&${ 9.69 }^{ +0.36 }_{ -0.40 }$&${ 879 }^{ +60 }_{ -76 }$&${ 0.15 }^{ +0.12 }_{ -0.10 }$&--&--&--\\ 
K00797.01 &${ 0.93 }^{ +0.05 }_{ -0.06 }$&${ 0.90 }^{ +0.10 }_{ -0.07 }$&${ 5613 }^{ +112 }_{ -134 }$&${ 4.51 }^{ +0.05 }_{ -0.10 }$&${ -0.08 }^{ +0.15 }_{ -0.19 }$&${ 9.61 }^{ +0.33 }_{ -0.39 }$&${ 1190 }^{ +137 }_{ -100 }$&${ 0.21 }^{ +0.09 }_{ -0.12 }$&--&--&--\\ 
K00798.01 &${ 0.89 }^{ +0.04 }_{ -0.04 }$&${ 0.85 }^{ +0.08 }_{ -0.05 }$&${ 5342 }^{ +68 }_{ -105 }$&${ 4.53 }^{ +0.04 }_{ -0.09 }$&${ 0.07 }^{ +0.15 }_{ -0.14 }$&${ 9.68 }^{ +0.37 }_{ -0.44 }$&${ 950 }^{ +88 }_{ -58 }$&${ 0.34 }^{ +0.05 }_{ -0.11 }$&--&--&--\\ 
K00800.01 &${ 1.11 }^{ +0.08 }_{ -0.07 }$&${ 1.15 }^{ +0.31 }_{ -0.12 }$&${ 6003 }^{ +99 }_{ -137 }$&${ 4.36 }^{ +0.08 }_{ -0.18 }$&${ 0.09 }^{ +0.15 }_{ -0.16 }$&${ 9.54 }^{ +0.16 }_{ -0.31 }$&${ 1651 }^{ +438 }_{ -192 }$&${ 0.23 }^{ +0.05 }_{ -0.11 }$&--&--&--\\ 
K00800.02 &${ 1.11 }^{ +0.09 }_{ -0.07 }$&${ 1.15 }^{ +0.29 }_{ -0.13 }$&${ 6005 }^{ +102 }_{ -134 }$&${ 4.37 }^{ +0.08 }_{ -0.17 }$&${ 0.09 }^{ +0.15 }_{ -0.15 }$&${ 9.52 }^{ +0.18 }_{ -0.30 }$&${ 1651 }^{ +404 }_{ -190 }$&${ 0.23 }^{ +0.06 }_{ -0.11 }$&--&--&--\\ 
K00801.01 &${ 1.11 }^{ +0.11 }_{ -0.07 }$&${ 1.16 }^{ +0.33 }_{ -0.13 }$&${ 5963 }^{ +94 }_{ -126 }$&${ 4.36 }^{ +0.09 }_{ -0.19 }$&${ 0.15 }^{ +0.14 }_{ -0.15 }$&${ 9.53 }^{ +0.17 }_{ -0.32 }$&${ 1145 }^{ +327 }_{ -137 }$&${ 0.53 }^{ +0.05 }_{ -0.10 }$&--&--&--\\ 
K00802.01 &${ 0.96 }^{ +0.09 }_{ -0.07 }$&${ 0.95 }^{ +0.17 }_{ -0.09 }$&${ 5731 }^{ +255 }_{ -188 }$&${ 4.47 }^{ +0.07 }_{ -0.12 }$&${ -0.06 }^{ +0.15 }_{ -0.19 }$&${ 9.57 }^{ +0.30 }_{ -0.35 }$&${ 1247 }^{ +237 }_{ -130 }$&${ 0.24 }^{ +0.21 }_{ -0.16 }$&--&--&--
\enddata
\tablecomments{A portion of this table is shown for form and content.  
                The full table will be available online.}
\end{deluxetable*}

\capstarttrue

\capstartfalse

\begin{deluxetable*}{cccccccccccccccc}
\rotate
\tablewidth{0pt}
\tabletypesize{\scriptsize}
\tablecaption{False Positive Probability Results
\tablabel{fpp}}
\tablehead{\colhead{KOI} &
    \colhead{$P$} &
    \colhead{$TTVs?$} &
    \colhead{$R_p$} &
    \colhead{$\delta_{\rm sec}$\tablenotemark{a}} &
    \colhead{$r_{\rm excl}$\tablenotemark{b}} &
    \colhead{Pr$_{\rm EB}$\tablenotemark{c}} &
    \colhead{Pr$_{\rm HEB}$\tablenotemark{c}} &
    \colhead{Pr$_{\rm BEB}$\tablenotemark{c}} &
    \colhead{$L_{\rm tot}$\tablenotemark{d}}&
    \colhead{$f_p$\tablenotemark{e}} &
    \colhead{$p_{\rm pos}$\tablenotemark{f}} &
    \colhead{$s_{\rm pos}$\tablenotemark{g}} &
    \colhead{Disp.\tablenotemark{h}} &
    \colhead{FPP\tablenotemark{i}} &
    \colhead{Failure\tablenotemark{j}} \\
    \colhead{} &
    \colhead{(d)} &
    \colhead{($R_\oplus$)} &
    \colhead{(ppm)} &
    \colhead{(\arcsec)} &
    \colhead{} &
    \colhead{} &
    \colhead{} &
    \colhead{} &
    \colhead{} &
    \colhead{} &
    \colhead{} &
    \colhead{} &
    \colhead{} &
    \colhead{} &
    \colhead{}
    }
\startdata
& 6.850 & False & 1.79 & 37 & 1.65 & 1e-05 & 1e-05 & 1e-05 & 1.2e+02 & 0.196 & 1.00 & 0.88 & PL & 1e-05 & -- \\ 
K02196.01 & 1.205 & False & 19.16 & 6 & 4.00 & 0.14 & 0.00082 & 0.84 & 51 & 0.017 & 0.04 & 1.00 & FP & 0.99 & -- \\ 
K02197.01 & 1.710 & False & 2.25 & 69 & 1.02 & 1e-05 & 1e-05 & 0.00071 & 1e-05 & 0.211 & 0.00 & 1.00 & FP & 0.00071 & -- \\ 
K02198.01 & 23.921 & False & 2.81 & 51 & 0.81 & 0.0047 & 2.4e-05 & 1e-05 & 1.8e+02 & 0.193 & 1.00 & 1.00 & CA & 0.0047 & -- \\ 
K02199.01 & 9.033 & False & 2.13 & 195 & 0.75 & 1e-05 & 1e-05 & 0.11 & 1.6 & 0.218 & 1.00 & 0.36 & CA & 0.11 & -- \\ 
K02199.02 & 3.053 & False & 1.42 & 120 & 1.56 & 1e-05 & 1e-05 & 3.6e-05 & 5.3e+02 & 0.135 & 0.61 & 0.11 & CA & 3.6e-05 & -- \\ 
K02200.01 & 3.171 & False & 2.12 & 87 & 1.62 & 0.00041 & 1e-05 & 0.0016 & 1.7e+03 & 0.214 & 1.00 & 1.00 & CA & 0.002 & -- \\ 
K02201.01 & 116.521 & False & 35.86 & 201 & 0.50 & 0.89 & 0.081 & 0.025 & 3 & 0.001 & 1.00 & 1.00 & CA & 1 & -- \\ 
K02202.01 & 0.813 & False & 1.19 & 24 & 0.50 & 1e-05 & 1e-05 & 0.00011 & 3.1e+03 & 0.117 & 1.00 & 1.00 & CA & 0.00011 & -- \\ 
K02203.01 & 1.990 & False & 1.62 & 27 & 1.20 & 1e-05 & 1e-05 & 5.9e-05 & 19 & 0.191 & 0.00 & 0.22 & FP & 5.9e-05 & -- \\ 
K02203.02 & 172.179 & False & 3.64 & 1170 & 5.40 & 0.14 & 0.082 & 0.65 & 2.5 & 0.098 & -- & -- & FP & 0.87 & -- \\ 
K02204.01 & 10.861 & False & 2.30 & 47 & 0.54 & 0.0076 & 0.00087 & 4.5e-05 & 3.6e+02 & 0.202 & 1.00 & 1.00 & CA & 0.0085 & -- \\ 
K02205.01 & 24.086 & False & 2.85 & 335 & 0.78 & 0.0043 & 0.00011 & 0.00062 & 1.8e+02 & 0.200 & 1.00 & 0.82 & CA & 0.005 & -- \\ 
K02206.01 & 1.834 & False & 1.62 & 81 & 4.00 & 1e-05 & 1e-05 & 0.012 & 4.6e+02 & 0.170 & -- & 0.00 & CA & 0.012 & -- \\ 
K02207.01 & 67.232 & False & 47.96 & 174 & 0.50 & 1e-05 & 1e-05 & 1 & 5.8e-05 & 0.001 & 0.00 & 0.50 & FP & 1 & -- \\ 
K02208.01 & 2.345 & False & 0.86 & 15 & 1.65 & 0.00035 & 1e-05 & 0.00043 & 71 & 0.077 & 1.00 & 1.00 & CA & 0.00078 & -- \\ 
K02209.01 & 18.303 & False & 1.36 & 40 & 1.11 & 1e-05 & 1e-05 & 0.0041 & 1.4 & 0.148 & 1.00 & 0.95 & CA & 0.0041 & -- \\ 
K02209.02 & 35.503 & False & 1.14 & 65 & 1.89 & 1e-05 & 1e-05 & 0.013 & 1.6 & 0.109 & 1.00 & 0.02 & CA & 0.013 & -- \\ 
K02210.01 & 2.889 & False & 1.93 & 576 & 0.84 & 1e-05 & 1e-05 & 7.6e-05 & 1.3e+02 & 0.196 & 1.00 & 0.99 & CA & 7.6e-05 & -- \\ 
K02210.02 & 210.631 & False & 3.68 & 79 & 1.14 & 1e-05 & 1e-05 & 3.9e-05 & 4.6 & 0.127 & 1.00 & 1.00 & CA & 3.9e-05 & -- \\ 
K02211.01 & 0.805 & False & 2.03 & 38 & 0.50 & 0.00078 & 1e-05 & 0.00011 & 2.4e+04 & 0.219 & 0.00 & 1.00 & FP & 0.00089 & -- \\ 
K02212.01 & 17.146 & False & 2.52 & 81 & 0.51 & 0.00042 & 1e-05 & 1.3e-05 & 1.9e+02 & 0.205 & 1.00 & 1.00 & CA & 0.00043 & -- \\ 
K02213.01 & 3.971 & False & 1.39 & 71 & 0.72 & 1e-05 & 1e-05 & 0.00014 & 1.8e+03 & 0.137 & 1.00 & 1.00 & CA & 0.00014 & -- \\ 
K02214.01 & 2.352 & False & 1.31 & 83 & 0.87 & 1e-05 & 1e-05 & 9.7e-05 & 2e+03 & 0.120 & 1.00 & 1.00 & CA & 9.7e-05 & -- \\ 
K02215.01 & 22.045 & False & 1.69 & 110 & 2.07 & 1e-05 & 1e-05 & 0.00036 & 11 & 0.168 & 0.92 & 1.00 & CA & 0.00036 & -- 
\enddata
\tablecomments{A portion of this table is shown for form and content.  
                The full table will be available online.}
\tablenotetext{a}{Maximum secondary eclipse depth allowed.}
\tablenotetext{b}{``Exclusion radius'' inside of which false positive scenarios are allowed.}
\tablenotetext{c}{Probabilities for different astrophysical false positive scenarios: 
                unblended eclipsing binary (EB), hierarchical eclipsing binary (HEB),
                and background/foreground eclipsing binary (BEB).}
\tablenotetext{d}{Sum of likelihood $\times$ prior for all models.  A very low
                    value here indicates that no model gives a good fit, so a low
                    FPP should be interpreted cautiously.}
\tablenotetext{e}{Assumed ``specific planet occurrence rate'' for this planet.}
\tablenotetext{f}{Probability of signal to be on target star, according to Bryson et al.~(2015, in prep).}
\tablenotetext{g}{Positional probability score, from Bryson et al. (2015).}
\tablenotetext{h}{Exoplanet Archive disposition: false positive (FP), candidate (CA), or confirmed (PL).}
\tablenotetext{i}{False positive probability.}
\tablenotetext{j}{Reason for failure: (1) No MCMC modeling available from \citet{Rowe:2015};
    (2)  Unphysical MCMC fit from \citet{Rowe:2015};
    (3)  No stellar parameters available from \citet{Huber:2014};
    (4)  No weak secondary data available;
    (5)  MCMC trapezoid fit did not converge;
    (6)  Period too short for implied star (orbit within star);
    (7)  Other unspecified \vespa\ error.}
\end{deluxetable*}

\capstarttrue


\end{document}

%%
%% End of file `sample.tex'.
